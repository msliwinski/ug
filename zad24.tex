
\documentclass[12pt,a4paper]{article}

% ustawienia marginesu
\usepackage[left=1.6in,right=.8in,top=1.5in,bottom=1.5in]{geometry}

% polskie reguły dzielenia wyrazów itd
\usepackage{polski}

% polskie znaki zakodowane w UTF8
\usepackage[utf8]{inputenc}

% automatyczne generowanie odnośników w plikach PDF
\usepackage[pdftex,linkbordercolor={0 0.9 1}]{hyperref}

% pakiety matematyczne
\usepackage{amsthm,amsmath,amsfonts,amssymb,mathrsfs}

\usepackage{ gensymb }

\usepackage{esint}

\usepackage{amssymb}

% ładne składanie odnośników do stron www
\usepackage{url}

% rozbudowane możliwości wypunktowań
\usepackage{enumerate}

% możliwość dodawania plików graficznych
\usepackage{graphicx} 

%%% definicje twierdzeń itd :)
\newtheorem{tw}{Twierdzenie}[section]
\newtheorem{stw}[tw]{Stwierdzenie}
\newtheorem{fakt}[tw]{Fakt}
\newtheorem{lemat}[tw]{Lemat}

\theoremstyle{definition}
\newtheorem{df}[tw]{Definicja}
\newtheorem{ex}[tw]{Przykład}
\newtheorem{uw}[tw]{Uwaga}
\newtheorem{wn}[tw]{Wniosek}
\newtheorem{zad}{Zadanie}

% oznaczenia zbiorów liczbowych
\DeclareMathOperator{\R}{\mathbb{R}}
\DeclareMathOperator{\Z}{\mathbb{Z}}
\DeclareMathOperator{\N}{\mathbb{N}}
\DeclareMathOperator{\Q}{\mathbb{Q}}


% wartość bezwzględna, norma, iloczyn skalarny, nośnik, rozpięcie przestrzeni...
\providecommand{\abs}[1]{\left\lvert#1\right\rvert}
\providecommand{\var}[1]{\operatorname{var}(#1)}

% fajne nagłówki i stopki na stronie
\usepackage{fancyhdr}
\pagestyle{fancy}
\fancyhf{}
\fancyfoot[R]{\textbf{\thepage}}
\fancyhead[L]{\small\sffamily \nouppercase{\leftmark}}
\renewcommand{\headrulewidth}{0.4pt} 
\renewcommand{\footrulewidth}{0.4pt}

% typowe dane dokumentu
\title{Funkcje ciągłe i różniczkowalne}
\date{\today}

% tu podaj swoje imię i nazwisko!
\author{Marcin Śliwiński}

% zaczynamy dokument
\begin{document}

\maketitle

\tableofcontents

\section{Funkcje ciagłe}
\begin{df}
(funkcja ciągła). Niech $f:(a,b)\rightarrow\R$, oraz niech $x_0\in(a,b)$. Mówimy, że funkcja $f$ jest ciągla w punkcie $x_0$ wtedy i tylko wtedy, gdy:
\[\forall_{\varepsilon>0}\exists_{\delta>0}\forall{x \in (a,b) |x-x_0|} < \delta \Rightarrow |f(x) - f(x_0)| < \varepsilon\]
\end{df}
\begin{ex}
Wielomiany, funkcje trygonometryczne, wykładnicze, logarytmiczne,  są ciągłe w każdym punkcie swojej dziedziny.
\end{ex}

\begin{ex}
Funkcja $f$ dana wzorem:
\begin{displaymath}
f(x)= \left\{ \begin{array}{ll}
x+1 & \textrm{dla $x\neq0$}\\
0 & \textrm{dla $x=0$}\\
\end{array} \right.
\end{displaymath}
Jest ciągła w każdym punkcie poza $x_0=0$.
\newline Niech $\Q$ oznacza zbiór wszystkich liczb wymiernych.
\end{ex}
\begin{ex}
Funkcja $f$ dana wzorem:
\begin{displaymath}
f(x)= \left\{ \begin{array}{ll}
0 & \textrm{dla $x\in\Q$}\\
1 & \textrm{dla $x\notin\Q$}\\
\end{array} \right.
\end{displaymath}
nie jest ciągła w żadnym punkcie.
\end{ex}
\begin{ex}
Funkcja $f$ dana wzorem:
\begin{displaymath}
f(x)= \left\{ \begin{array}{ll}
0 & \textrm{dla $x\in\Q$}\\
x & \textrm{dla $x\notin\Q$}\\
\end{array} \right.
\end{displaymath}
jest ciągła w punkcie $x_0=0$, ale nie jest ciągła w pozostałych punktach dziedziny.
\end{ex}

\begin{zad}
Udowonij prawdziwość podanychprzykładów.
\end{zad}

\begin{df}
Jeśli funkcja $f:A\rightarrow\R$ jest ciągła w każdym punkcie swojej dziedziny $A$ to mowimy krótko, że jest ciągła.
\end{df}

\begin{flushright}
Ponizsze twierdzienie zbiera podstawowe własności zbioru funkcji ciągłych.
\end{flushright}

\begin{tw}
\textit{Niech funkcje $f,g: R\rightarrow\R$ będą ciągłe, oraz niech $\alpha,\beta\in\R$}

\textit{Wtedy Funkcje:}
\begin{enumerate}[a)]
\item $h_1(x)=\alpha\cdot f(x)+\beta\cdot g(x),$
\item $h_2(x)=f(x)\cdot g(x),$
\item $h_3(x)=\frac{f(x)}{g(x)}$ \textit{(o ile $g(x)\neq 0$ dla dowolnego $x\in\R$),}
\item $h_4(x)=f(g(x)),$
\end{enumerate}
\textit{są ciągłe.}

\end{tw}

\	Następne twierdzenie zwane powszechnie "wlasnością Darboux" lub twierdzeniem o wartości pośredniej ma liczne praktyczne zastosowania, Mowi ono o tym,
że jeśli funkcja ciągła przyjmuje jakieś dwie wartości, to przy odpowiednich założeniach co do dziedziny, przyjmuje też wszystkie warotści pośrednie.
Możemy sobie to łatwo wyobraxić na przykładzie funkcji, która opisuje  zmianę temperatury w czasie. Jeśli o 7:00 było $-1\celsius$ a o 9:00 było $2\celsius$,
to zapewme gdzieś między 7:00 a 9:00 był taki moment, że temperatura wynosiła dokładnie $0\celsius$.

\begin{tw}
\textit{Niech $f:[a,b]\rightarrow\R$ ciągła, oraz niech $f(a)\neq f(b)$.
Wtedy dla dowolnego $y_0\in conv\{f(a),f(b)\}$ istnieje $x_0\in [a,b]$ takie, że $f(x_0)=y_0$.}
\end{tw}

\section{Różniczkowalność}
\begin{df}
Niech $f:(a,b)\rightarrow\R$, $x_0\in (a,b)$ oraz $f$ ciągła w otoczeniu punktu $x_0$. Jeśli istnieje granica:
\begin{displaymath}
\lim_{n \to \infty}
\frac{f(x)-f(x_0)}{x-x_0}
\end{displaymath}
i jest skończona, to oznaczmy ją przez $f'(x_0)$ i nazywamy pochodną funkcji $f$ w punkcie $x_0$.
\end{df}

\end{document}
